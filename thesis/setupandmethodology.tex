% !TEX encoding = UTF-8 Unicode
%!TEX root = thesis.tex
% !TEX spellcheck = en-US
\chapter[Setup and Methodology]{Setup and Methodology}

\section{Test platfoms}

\subsection{Arendale Duo}
The Arendale Duo board was used for some perliminary research in this thesis.
It was chosen because we had experience from earlier student projects using this board.
It's feature set and properties are elaborated in section \fullref{ArendaleBoard}.
\begin{table}[h]
  \begin{tabular}{ll}
    \textbf{SoC}              & Samsung Exynox 5250 \\
    \textbf{CPU}              &  \\
    Model                     & ARM Cortex-A15 \\
    Manufacturing process     & 32nm \\
    Maxiumu clock frequency   & 1.7GHz \\
    Number of cores           & 2 \\
    L2 Cache                  & 1MB \\
    L1 Cache                  & 32KB \\
    \textbf{GPU}              &  \\
    Model                     & ARM Mali-T604 \\
    Maxiumu clock frequency   & 600 MHz \\
    Number of cores           & 4 \\
    \textbf{Memory}           &  \\
    Available memory          & 2 GB \\
    Maxiumu clock frequency   & 800MHz \\
    \textbf{Operating system} &  \\
    Distriubtion              & Linux Ubuntu \\
    Version                   & TODO
  \end{tabular}
  \caption{Arendale Duo Spesifications\label{overflow}}
\end{table}
\subsection{ODROID-XU3}
The ODROID-XU3 single board computing system was used for most of the experiments of this thesis.
Fitted with an Samsung Exynos 5422 SoC it offer the heterogenous properties that will be explored in detail in the planned master thesis.
In adition, the board offer multiple energy monitors, enabling precise data gathering.
It's feature set and properties are elaborated in section \fullref{OdroidXU3}.

\begin{table}[h]
  \begin{tabular}{ll}
    \textbf{SoC}              & Samsung Exynox 5422 \\
    \textbf{CPU 1}            &  \\
    Model                     & ARM Cortex-A15 \\
    Manufacturing process     & 32nm \\
    Maxiumu clock frequency   & 2.0GHz \\
    Number of cores           & 4 \\
    L2 Cache                  & 512KB \\
    L1 Cache                  & 32KB/32KB I/D \\
    \textbf{CPU 2}            &  \\
    Model                     & ARM Cortex-A7 \\
    Manufacturing process     & 32nm \\
    Maxiumu clock frequency   & 1.4GHz \\
    Number of cores           & 4 \\
    L2 Cache                  & 2MB \\
    L1 Cache                  & 32KB/32KB I/D \\
    \textbf{GPU}              &  \\
    Model                     & ARM Mali-T628 MP6 \\
    Maxiumu clock frequency   & 600 MHz \\
    Number of cores           & 4 \\
    \textbf{Memory}           &  \\
    Available memory          & 2 GB \\
    Maxiumu clock frequency   & 933MHz \\
    \textbf{Operating system} &  \\
    Distriubtion              & Linux odroid \\
    Version                   & 3.10.54+
  \end{tabular}
  \caption{ODROID-XU3 Spesifications\label{overflow}}
\end{table}

%\subsection{Comparison of the test benches}

\section{Software}
\begin{table}[H]
  \begin{tabular}{ll}
    \textbf{Software} & \textbf{Version}  \\
    Nanos++           & 0.9a              \\
    Mercurium         & 1.99.4            \\
    Papi              & 5.3.2             \\
    Extrae            & 3.0.1             \\
    gcc               & 4.8.2             
  \end{tabular}
  \caption{Third party software and frameworks used in the experiments.\label{overflow}}
\end{table}


\section{Compilation and running of test benches}
Mention frequency scaling here.

\section{Performance measurment}
In the experiments being run in this pilot project, execution time was used as the primary metric for performance.
While running the experiments, the POSIX function gettimeofday() is used before and after running the main part of the application.
The time difference is a used as a measurement of how good the performance is.

In the planned master thesis, PAPI (Performance API) may be used to measure more detailed aspects of the performance.
PAPI is able to access alot of different performance measurements from systems.
The avaiable metrics include among others; Cache utilization and hit rate, memory utilization and bus utilization.

\section{Energy efficiency measurement}
Because of the different features of the two experiment platforms, two different energy efficiency measurement schemes was used.

\subsection{ODROID-XU3}
As elaborated in section \fullref{OdroidXU3}, the ODROID-XU3 feature 4 current sensors.
These current sensors are used to monitor the current flow to the large CPU, small CPU, GPU and memory respectivly.
In addition to the current sensors, it is possible to read the supply voltage to each of these components.
Based on these two sensor readings, we can calculate the power consumption of each of the components in realtime while running our applications.
Using this scheme for power measurement, we are able to get readings with high precission of each component.
The results will show how the consumption vary with different application configurations both for the application as a whole and for different stages of execution.

$ Power\ consumption (Watt) = Current (Ampere) \times Voltage (Volt) $

\subsection{Arendale Duo}

