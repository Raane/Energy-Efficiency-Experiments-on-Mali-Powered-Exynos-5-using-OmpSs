% !TEX encoding = UTF-8 Unicode
%!TEX root = thesis.tex
% !TEX spellcheck = en-US
\chapter[Future Work]{Future Work}
These are some suggestions for future work that may build upon the work in this thesis.

\section{Experiment with heterogeneity}
In this thesis, there have been done experiments with the Exynos 5, which support ARM big.LITTLE.
The heterogeneous properties of this processor was outside of the scope of this pilot project.
The same applications can be adapted and optimized to explore the potential of this processor architecture.
This is planned for the master thesis following this pilot project.

\section{OmpSs with OpenCL kernels}
A new feature of OmpSs is it's ability to manage OpenCL kernels as tasks.
It is possible to issue OpenCL kernels as OmpSs tasks, and have the task manager assign them to GPUs and CPUs.
This allow for portable code that can run effectively on a range of different system.
It would be interesting to examine the potency of this way of utilizing the GPU, as it save the programmer from the job of manually tuning the load balance between GPU and CPU.

\section{ARMv8-A 64-bit processors}
ARM have created the next generation ARM processors.
They run a new instruction set, with support for both 32- and 64-bit instructions.
Running similar experiments on such a processor would be interesting.

\section{Performance measurement}
In this project, only simple forms of performance measurement was used.
Running the same or similar experiments with access to other performance counters would be interesting.
Cache hit rate and utilization, memory utilization, bus utilization and other counters could help understanding the strengths and weaknesses of the system.

