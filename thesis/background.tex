% !TEX encoding = UTF-8 Unicode
%!TEX root = thesis.tex
% !TEX spellcheck = en-US
\chapter[Background]{Background}

\section{Energy measurement}

\section{NEON}

\section{Task based programming}
Task based programming allow a programmer to work with parallel programs, with an abstraction from the program itself.
When programming with this model, the program can be split into tasks which can run in parallel.
When the program run, it will run a task manager as part of the program.
This task manager can dynamically assign tasks to the processors, and the programmer does not have to handle all the time consuming tasks related to manual parallelisation.

The task based programming model also allow simpler development of protable programs.
When the program is running tasks on available CPUs, it is not a problem to allow it to run on larger or smaller numbers of processors, and even clusters can support the program.
This model even allow the tasks to run on different types of processors in a hetrogenous enviroment   

\section[OmpSs]{OpenMP Super scalar}
OpenMP Super scalar is a extention of the OpenMP API to integrate features from the StarSs programming model.
It is currently under development at the Barcelona Supercomputing Center.
The goal of OmpSs is to extend the programming model to support a wide range og processors.
The OmpSs programming model will run on a wide variety of different systems, such as traditional personal computers, clusters, shared memory systems and hetrogenous processors.
While the software is not yet comlpeted or fully tested, there have been several reports exploring it's potentilal.
The results have proven OmpSs as an efficient solution on both clusters and hetrogenous systems utilizing OpenCL and CUDA.


\section{Heterognous multi-processor}
Heterognous multi-processor systems have multiple different processors, opposed to traditional multi-processor systems.
A typical modern processor have several processors, and a program can run effectivly by having threads running parts of theis work on each of them.
This work is often of such a nature that it can run better on a different processor.
Sometimes it can run just as well on multiple simple processor, while using less die space and energy.
In other instances, an advanced processor with some special capabilities, like vector instructions, can be more efficient.

  This kind of processors have a potential to help us overcome the challenges that are emerging in processor development.
    Unfortunatly they also introduce several new challenges.


\section{Arendale Board}
\subsection{ARM Cortex-A15}
\subsection{ARM Mali T604}

\section{Odroid}
\subsection{ARM Cortex-A15}
\subsection{ARM Cortex-A7}
\subsection{ARM Mali T628}

\section{Algorithms}
Here I will write about the algorithms used in the experiments.

